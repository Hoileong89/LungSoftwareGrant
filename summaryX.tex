\documentclass[11pt,]{article}
\usepackage{lmodern}
\usepackage{amssymb,amsmath}
\usepackage{ifxetex,ifluatex}
\usepackage{fixltx2e} % provides \textsubscript
\ifnum 0\ifxetex 1\fi\ifluatex 1\fi=0 % if pdftex
  \usepackage[T1]{fontenc}
  \usepackage[utf8]{inputenc}
\else % if luatex or xelatex
  \ifxetex
    \usepackage{mathspec}
    \usepackage{xltxtra,xunicode}
  \else
    \usepackage{fontspec}
  \fi
  \defaultfontfeatures{Mapping=tex-text,Scale=MatchLowercase}
  \newcommand{\euro}{€}
    \setmainfont{Georgia}
\fi
% use upquote if available, for straight quotes in verbatim environments
\IfFileExists{upquote.sty}{\usepackage{upquote}}{}
% use microtype if available
\IfFileExists{microtype.sty}{%
\usepackage{microtype}
\UseMicrotypeSet[protrusion]{basicmath} % disable protrusion for tt fonts
}{}
\usepackage[margin=0.5in]{geometry}
\ifxetex
  \usepackage[setpagesize=false, % page size defined by xetex
              unicode=false, % unicode breaks when used with xetex
              xetex]{hyperref}
\else
  \usepackage[unicode=true]{hyperref}
\fi
\hypersetup{breaklinks=true,
            bookmarks=true,
            pdfauthor={},
            pdftitle={},
            colorlinks=true,
            citecolor=blue,
            urlcolor=blue,
            linkcolor=magenta,
            pdfborder={0 0 0}}
\urlstyle{same}  % don't use monospace font for urls
\setlength{\parindent}{0pt}
\setlength{\parskip}{6pt plus 2pt minus 1pt}
\setlength{\emergencystretch}{3em}  % prevent overfull lines
\setcounter{secnumdepth}{0}

%%% Use protect on footnotes to avoid problems with footnotes in titles
\let\rmarkdownfootnote\footnote%
\def\footnote{\protect\rmarkdownfootnote}

%%% Change title format to be more compact
\usepackage{titling}

% Create subtitle command for use in maketitle
\newcommand{\subtitle}[1]{
  \posttitle{
    \begin{center}\large#1\end{center}
    }
}

\setlength{\droptitle}{-2em}
  \title{}
  \pretitle{\vspace{\droptitle}}
  \posttitle{}
  \author{}
  \preauthor{}\postauthor{}
  \date{}
  \predate{}\postdate{}

\usepackage{booktabs}
\usepackage[font={small},labelfont=bf,labelsep=colon]{caption}
\linespread{0.90}
\usepackage[compact]{titlesec}
\usepackage{enumitem}
\usepackage{tikz}
\def\checkmark{\tikz\fill[scale=0.4](0,.35) -- (.25,0) -- (1,.7) -- (.25,.15) -- cycle;}
\setlist{nolistsep}
\titlespacing{\section}{2pt}{*0}{*0}
\titlespacing{\subsection}{2pt}{*0}{*0}
\titlespacing{\subsubsection}{2pt}{*0}{*0}
\setlength{\parskip}{3pt}


\begin{document}

\maketitle


\pagenumbering{gobble}

\begin{center}
{\bf Summary}
\end{center}

The advent of reproducibility and other open-science concerns have
highlighted the importance of publicly available data as well as the
necessary tools for processing and analysis. This is certainly the case
in the neuroscience community where several publicly available software
packages for imaging (e.g., SPM, FSL, AFNI, ANTs) have been heavily
utilized directly resulting in significant research and progress within
the field. These packages contain essential application-specific
programs for a variety of neuroimaging tasks such as brain extraction,
$n$-tissue brain segmentation, and multi-atlas cortical labeling for
producing discriminative biomarkers for clinical exploration (e.g.,
cortical thickness). However, despite the obvious benefits that such
tools have within their respective communities, there are no analogous
packages for the pulmonary imaging community where relevant programs
would address such functionality as lung extraction, lobe and airway
segmentation, and kinematic inference via image registration.

ITK-Lung brings together leading expertise in pulmonary image analysis
at Penn and the University of Virginia to develop, evaluate, and deploy
a critical open-science resource under community support which would
allow access to multi-modal data and tools for processing and analysis
of lung data. Toward this end, a comprehensive image analysis and data
package, denoted as ITK-Lung, will be developed for the pulmonary
imaging community. This first-of-its-kind package will be enhanced by
data-specific tuning that will accommodate a variety of user backgrounds
and needs. These developments will be evaluated through their
application to real-world use cases and their practical integration with
major community resources. The successful completion of this project
will fully realize the value of ITK-Lung in pulmonary research and lead
to an immediate and broad impact on the field.

\end{document}
