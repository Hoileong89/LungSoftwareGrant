
\begin{table}[!t]
  \small
   \centering
    \begin{tabular*}{\textwidth}{c @{\extracolsep{\fill}} ccc}
    \toprule
    {\bf Functionality} & {\bf CT} & {\bf 1H MRI} & {\bf 3He MRI}\\
    \cmidrule[1pt](lr){1-4}
    spatial normalization & { $\bullet$ } & { $\bullet$ } & { $\bullet$ } \\
    template generation & { $\bullet$ } & { $\bullet$ } & { $\bullet$ } \\
    lung segmentation & { $\bullet$ } & { $\bullet$ } & {  } \\
    lobe segmentation & { $\bullet$ } & { $\bullet$ } & {  } \\
    airway segmentation & { $\bullet$ } & { } & {  } \\
    vessel segmentation & { $\bullet$ } & { } & {  } \\
    functional segmentation & { $\bullet$ } & {  } & { $\bullet$ } \\
    nodule detection & { $\bullet$ } & {  } & {  } \\
    feature indices & { $\bullet$ } & {  } & { $\bullet$ } \\
    \bottomrule
   \end{tabular*}
 \label{table:algorithms}
 \caption{Specific outline of basic functionality offered with this proposal
 categorized by modality.  One of the motivations for the collaborative use
 cases as a specific aim is the inevitability that other lung-specific
 algorithmic needs will be identified and will be added to the functionality
 developed and offered as part of this proposal.  It should also be noted that
 some modality-specific modifications will be required.  For example, although
 our lobe estimation approach works well for 1H MRI where no internal anatomical
 features are available for refinement, the same is not true
 for CT.  However, this lobe estimation strategy can be applied to CT in providing
 spatial prior maps for subsequent subject-specific refinement.
 }

\end{table}
