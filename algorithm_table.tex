
\begin{table}[!t]
  \small
   \centering
    \begin{tabular*}{0.75\textwidth}{c @{\extracolsep{\fill}} cccc}
    \toprule
    {\bf Functionality} & {\bf CT} & {\bf 1H MRI} & {\bf 3He MRI} & {\bf PET}\\
    \cmidrule[1pt](lr){1-5}
    registration and normalization & { \checkmark } & { \checkmark } & { \checkmark } & {}\\
    template generation & { \checkmark } & { \checkmark } & { \checkmark } & {} \\
    lung segmentation & { \checkmark } & { \checkmark } & {  } & {} \\
    lobe segmentation & { \checkmark } & { \checkmark } & {  } & {} \\
    airway segmentation & { \checkmark } & { } & {  } & {} \\
    vessel segmentation & { \checkmark } & { } & {  } & {} \\
    functional segmentation & { \checkmark } & {  } & { \checkmark }  & {}\\
    feature indices & { \checkmark } & {  } & { \checkmark }  & {}\\
    \bottomrule
   \end{tabular*}
 \label{table:algorithms}
 \caption{Core functionality proposed for development and evaluation
 in the project
 categorized by modality.  One of the motivations for the collaborative use
 cases as a specific aim is the inevitability that other lung-specific
 algorithmic needs will be identified and will be added to the functionality
 developed and offered as part of this project.  It should also be noted that
 some modality-specific modifications will be required.  For example,
 our lobe estimation approach works well for 1H MRI where no internal anatomical
 features are available for refinement.  This lobe estimation strategy can be directly applied to CT in providing
 spatial prior maps for subsequent subject-specific refinement.
 }

\end{table}
