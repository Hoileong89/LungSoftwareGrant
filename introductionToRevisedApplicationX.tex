\documentclass[11pt,]{article}
\usepackage{lmodern}
\usepackage{amssymb,amsmath}
\usepackage{ifxetex,ifluatex}
\usepackage{fixltx2e} % provides \textsubscript
\ifnum 0\ifxetex 1\fi\ifluatex 1\fi=0 % if pdftex
  \usepackage[T1]{fontenc}
  \usepackage[utf8]{inputenc}
\else % if luatex or xelatex
  \ifxetex
    \usepackage{mathspec}
    \usepackage{xltxtra,xunicode}
  \else
    \usepackage{fontspec}
  \fi
  \defaultfontfeatures{Mapping=tex-text,Scale=MatchLowercase}
  \newcommand{\euro}{€}
    \setmainfont{Georgia}
\fi
% use upquote if available, for straight quotes in verbatim environments
\IfFileExists{upquote.sty}{\usepackage{upquote}}{}
% use microtype if available
\IfFileExists{microtype.sty}{%
\usepackage{microtype}
\UseMicrotypeSet[protrusion]{basicmath} % disable protrusion for tt fonts
}{}
\usepackage[margin=0.5in]{geometry}
\ifxetex
  \usepackage[setpagesize=false, % page size defined by xetex
              unicode=false, % unicode breaks when used with xetex
              xetex]{hyperref}
\else
  \usepackage[unicode=true]{hyperref}
\fi
\hypersetup{breaklinks=true,
            bookmarks=true,
            pdfauthor={},
            pdftitle={},
            colorlinks=true,
            citecolor=blue,
            urlcolor=blue,
            linkcolor=magenta,
            pdfborder={0 0 0}}
\urlstyle{same}  % don't use monospace font for urls
\setlength{\parindent}{0pt}
\setlength{\parskip}{6pt plus 2pt minus 1pt}
\setlength{\emergencystretch}{3em}  % prevent overfull lines
\providecommand{\tightlist}{%
  \setlength{\itemsep}{0pt}\setlength{\parskip}{0pt}}
\setcounter{secnumdepth}{0}

%%% Use protect on footnotes to avoid problems with footnotes in titles
\let\rmarkdownfootnote\footnote%
\def\footnote{\protect\rmarkdownfootnote}

%%% Change title format to be more compact
\usepackage{titling}

% Create subtitle command for use in maketitle
\newcommand{\subtitle}[1]{
  \posttitle{
    \begin{center}\large#1\end{center}
    }
}

\setlength{\droptitle}{-2em}
  \title{}
  \pretitle{\vspace{\droptitle}}
  \posttitle{}
  \author{}
  \preauthor{}\postauthor{}
  \date{}
  \predate{}\postdate{}

\usepackage{booktabs}
\usepackage[font={small},labelfont=bf,labelsep=colon]{caption}
\linespread{0.82}
\usepackage[compact]{titlesec}
\usepackage{enumitem}
\usepackage{tikz}
\def\checkmark{\tikz\fill[scale=0.4](0,.35) -- (.25,0) -- (1,.7) -- (.25,.15) -- cycle;}
\setlist{nolistsep}
\titlespacing{\section}{2pt}{*0}{*0}
\titlespacing{\subsection}{2pt}{*0}{*0}
\titlespacing{\subsubsection}{2pt}{*0}{*0}
\setlength{\parskip}{3pt}
\hypersetup{breaklinks=true, bookmarks=true, pdfauthor={}, pdftitle={}, colorlinks=true, citecolor=black, urlcolor=black, linkcolor=black, pdfborder={0 0 0}}

% Redefines (sub)paragraphs to behave more like sections
\ifx\paragraph\undefined\else
\let\oldparagraph\paragraph
\renewcommand{\paragraph}[1]{\oldparagraph{#1}\mbox{}}
\fi
\ifx\subparagraph\undefined\else
\let\oldsubparagraph\subparagraph
\renewcommand{\subparagraph}[1]{\oldsubparagraph{#1}\mbox{}}
\fi

\begin{document}
\maketitle

\pagenumbering{gobble}

\section{1. Introduction to the Revised
Application}\label{introduction-to-the-revised-application}

We would like to thank the reviewers for a careful critique, the resume
and summary of which commented positively on the ``potential high
significance'' of the ``excellent application'' as well as the
``expertise {[}of the team{]} in {[}both{]} lung imaging and software
development.'' We are equally grateful for the reviewers' helpful
suggestions, all of which were incorporated in this resubmission as
described below.\footnote{Changes to the Research Plan in response to
  reviewers' concerns are identified by vertical sidebars.}

\textbf{Lack of details for hardware, data, and method descriptions
(Critiques 1 and 2). } We agree and have expanded descriptions for all
elements of the project, including new discussion concerning GPU
development, and hardware platforms employed and, where applicable,
publicly available competition data for performance benchmarking. We
have also removed ``trivial'' items as per reviewer suggestions.

\textbf{Lack of PET imaging (Critiques 1 and 3).} We agree concerning
the importance of PET imaging in the lung and have now included this
modality in the project, as well as a related contribution in pulmonary
blood perfusion quantification based on recent work from our group.

\textbf{Dynamic nature of the lung environment and heterogeneity of
applications/equipment not addressed (Critiques 1 and 3).} The unique
dynamic nature of the multi-modality lung imaging environment indeed
poses significant challenges to acquiring high quality images of the
lung. Although acquisition protocols continue to improve---in part due
to precisely the kinds of image registration/processing innovations that
we propose to make available through ITK-Lung, the current
state-of-the-art is sufficient to allow effective quantitation of lung
structural and functional parameters in multi-modality studies. The
major caveat, however, is that sophisticated and extensive pre- and
post-processing of the images may be required depending on the type and
degree of distortions and artifacts introduced during acquisition.
Enabling such preprocessing is exactly one of the major goals of this
project. To achieve this objective, given the additional complexity
introduced, as noted in critique, by the heterogeneity of applications
and equipment in lung imaging, flexible and tunable (i.e., open and
programmable) tools are needed, with manifold capabilities carefully
curated to cover essential analysis and processing tasks, all of which
ideally integrated within a single coherent toolbox---this is precisely
the overall goal and deliverable of the proposed project.

\textbf{Excessive number of external collaborators (Critiques 1 and 2).}
It was suggested that the number of proposed external collaborators is
potentially excessive and possibly redundant. We agree with this
assessment and have limited the primary beta testers to research groups
at the project sites (i.e., the University of Pennsylvania and the
University of Virginia). However, we plan to maintain the proposed
relationships with the external groups but will scale back the direct
assistance to them. The use of separate, independent testing sites will
increase the value of the tools produced by ensuring that their success
is not specific to the particular source of data. This will increase the
generality of the developed resources and thus ease dissemination to the
wider community.

\textbf{Lack of novelty (Critiques 2 and 3).} Indeed, this project does
not seek to invent new algorithms and is motivated instead by the lack
of access to methods that define the current state-of-the-art, in
particular, within a comprehensive, integrated framework. Thus, in spite
of the relatively modest degree of algorithmic innovation, we believe
that significant novelty lies in the proposed ``initiation of a new
general toolkit focused on lung imaging'' as noted in Critique 1.

\textbf{Preexisting lung image analysis software (Critique 2).} Already
existing softwares were mentioned as obviating factors for the proposed
project. It is true that ``investigators have been assessing pulmonary
images of tumors/nodules, interstitial disease, and pulmonary embolism
for decades'' yet public availability of such work continues to be a
critical problem. Specific mention was made of available software
accompanying the LIDC and RIDER lung imaging databases. However, the
only software that is publicly accessible appears to be limited to
interpretation and translation of the XML files describing the nodule
annotations of those specific databases. In contrast, the COPDGene study
has given rise to the Chest Imaging Platform software, which does not
yet appear to be fully available to the public and whose focus and scope
(CT imaging primarily of COPD) are significantly more narrow than those
for this project. Similarly, we looked at the various lung-based image
processing competitions that have been held over the years and, to the
best of our knowledge, the vast majority of proposed algorithms are not
publicly available. Critique asserts that ``many of the feature indices
(Table 2) already exist in generic form in Brain packages, etc;''
however, we are unaware of such availability in the major brain packages
(e.g., FSL, SPM, FreeSurfer), although we do know that a number of them
exist in the Insight Toolkit because our group developed and contributed
the corresponding software. This project would target these available
features---as well as create additional ones specialized---to lung
images.

\textbf{Naivety on the part of the investigators (Critique 2).} The
concern is raised regarding the necessary level of sophistication of the
investigators to fulfill the project deliverables. Based on our
publication record in the lung imaging field (now documented in the
revised application), we respectfully submit that this assessment
represents a significant undervaluing of our contributions, where much
of our showcased ``preliminary'' software has been used in large
studies. A new letter from Eric Hoffman, an acknowledged authority in
the field, attests to our group's reputation in lung image analysis
methods and software development. Moreover, Dr.~Hoffman has joined the
project team, further strengthening feasibility of the project.

\textbf{Reduction of algorithmic scope (Critique 2).} Concerns
describing the scope of the project as being too large were raised in
light of the extensive methodological depth of the field. We agree and
have revised our application to sharpen the delineation of our
contributions.

\textbf{Budget (Critique 1).} This has been reduced as per review
suggestions and includes further cuts to reflect the more limited
support of external collaborators in the revised project.

\hypertarget{refs}{}

\end{document}
