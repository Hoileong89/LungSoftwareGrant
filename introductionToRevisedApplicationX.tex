\documentclass[11pt,]{article}
\usepackage{lmodern}
\usepackage{amssymb,amsmath}
\usepackage{ifxetex,ifluatex}
\usepackage{fixltx2e} % provides \textsubscript
\ifnum 0\ifxetex 1\fi\ifluatex 1\fi=0 % if pdftex
  \usepackage[T1]{fontenc}
  \usepackage[utf8]{inputenc}
\else % if luatex or xelatex
  \ifxetex
    \usepackage{mathspec}
    \usepackage{xltxtra,xunicode}
  \else
    \usepackage{fontspec}
  \fi
  \defaultfontfeatures{Mapping=tex-text,Scale=MatchLowercase}
  \newcommand{\euro}{€}
    \setmainfont{Georgia}
\fi
% use upquote if available, for straight quotes in verbatim environments
\IfFileExists{upquote.sty}{\usepackage{upquote}}{}
% use microtype if available
\IfFileExists{microtype.sty}{%
\usepackage{microtype}
\UseMicrotypeSet[protrusion]{basicmath} % disable protrusion for tt fonts
}{}
\usepackage[margin=0.5in]{geometry}
\ifxetex
  \usepackage[setpagesize=false, % page size defined by xetex
              unicode=false, % unicode breaks when used with xetex
              xetex]{hyperref}
\else
  \usepackage[unicode=true]{hyperref}
\fi
\hypersetup{breaklinks=true,
            bookmarks=true,
            pdfauthor={},
            pdftitle={},
            colorlinks=true,
            citecolor=blue,
            urlcolor=blue,
            linkcolor=magenta,
            pdfborder={0 0 0}}
\urlstyle{same}  % don't use monospace font for urls
\setlength{\parindent}{0pt}
\setlength{\parskip}{6pt plus 2pt minus 1pt}
\setlength{\emergencystretch}{3em}  % prevent overfull lines
\providecommand{\tightlist}{%
  \setlength{\itemsep}{0pt}\setlength{\parskip}{0pt}}
\setcounter{secnumdepth}{0}

%%% Use protect on footnotes to avoid problems with footnotes in titles
\let\rmarkdownfootnote\footnote%
\def\footnote{\protect\rmarkdownfootnote}

%%% Change title format to be more compact
\usepackage{titling}

% Create subtitle command for use in maketitle
\newcommand{\subtitle}[1]{
  \posttitle{
    \begin{center}\large#1\end{center}
    }
}

\setlength{\droptitle}{-2em}
  \title{}
  \pretitle{\vspace{\droptitle}}
  \posttitle{}
  \author{}
  \preauthor{}\postauthor{}
  \date{}
  \predate{}\postdate{}

\usepackage{booktabs}
\usepackage[font={small},labelfont=bf,labelsep=colon]{caption}
\linespread{0.92}
\usepackage[compact]{titlesec}
\usepackage{enumitem}
\usepackage{tikz}
\def\checkmark{\tikz\fill[scale=0.4](0,.35) -- (.25,0) -- (1,.7) -- (.25,.15) -- cycle;}
\setlist{nolistsep}
\titlespacing{\section}{2pt}{*0}{*0}
\titlespacing{\subsection}{2pt}{*0}{*0}
\titlespacing{\subsubsection}{2pt}{*0}{*0}
\setlength{\parskip}{3pt}

% Redefines (sub)paragraphs to behave more like sections
\ifx\paragraph\undefined\else
\let\oldparagraph\paragraph
\renewcommand{\paragraph}[1]{\oldparagraph{#1}\mbox{}}
\fi
\ifx\subparagraph\undefined\else
\let\oldsubparagraph\subparagraph
\renewcommand{\subparagraph}[1]{\oldsubparagraph{#1}\mbox{}}
\fi

\begin{document}
\maketitle

\pagenumbering{gobble}

\section{1. Introduction to the Revised
Application}\label{introduction-to-the-revised-application}

The initial application was reviewed by the BMIT-B study section and
scored in the 37th percentile. The resume and summary of the review
discussion commented positively on the potential high impact of the
``excellent application'' as well as the expertise of the team in both
lung imaging and software development. Concerns were discussed regarding
the lack of certain project details (vis-a-vis both hardware and data)
and the expected application areas. Additionally, issues were raised
concerning the dynamic nature of the lung and the related potential
difficulties with the proposed project. This revised application
addresses these concerns and includes additional details to motivate
reconsideration.

\textbf{Lack of details for hardware, data, and method descriptions.} We
agree and have added text detailing hardware requirements for the
proposed project including additional discussion concerning the GPU
hardware component. We have also expanded our proposed documentation
deliverables to include hardware platforms employed and benchmarking of
computational times and, where applicable, reporting of performance
based on publicly available competition data (Critique 1). Also, we have
expanded descriptions for the methodologies proposed in the revised
application and removed items that were deemed ``trivial.''

\textbf{Lack of PET imaging.} We agree concerning the importance of PET
imaging in the lung and have included discussion of possible data and
algorithmic contributions to be included in this project. Based on
recent work done by our group, we have added a related contribution
concerning quantification of pulmonary blood perfusion parameters which
we have added to this revised application (Critique 1).

\textbf{Lung motion not addressed.} \emph{Jim to add.}

\textbf{Excessive number of external collaborators (Critique 2).} It was
suggested that the number of proposed external collaborators is
potentially excessive and possibly redundant. We agree with this
assessment and have limited the number of beta testers to research
groups at the core institutions (i.e., the University of Pennsylvania
and the University of Virginia). As the other groups continue to have an
interest in the outcome of this project, we plan to maintain the
proposed relationships but intend to scale back the direct assistance
for individual projects that we had originally planned.

\textbf{Lack of novelty (Critiques 2 and 3).} Despite the admitted lack
of \emph{algorithmic} innovation, we believe that significant novelty
lies in the fact that an open-source package for lung image analysis has
yet to be offered which is analogous in functionality to critically
important brain packages. As we pointed out in the original application,
according to a prominent figure in pulmonary image research, such lack
has been one of the ``major hinderances to more widespread usage of such
{[}lung{]} imaging biomarkers.''

\textbf{Preexisting lung image analysis software (Critique 2).} Already
existing softwares were mentioned as obviating factors for the proposed
project. It is true that ``investigators have been assessing pulmonary
images of tumors/nodules, interstitutal disase, and pulmonary embolism
for decades'' yet public availability of such work continues to be a
critical problem. Specific mention was made of available software
accompanying the LIDC and RIDER lung imaging databases. However, the
only software we found were limited to interpretation and translation of
the XML files describing the nodule annotations of those specific
databases. Such software certainly does not comprise a part of the
generic processing utilities that we envision. Similarly, we looked at
the various lung-based image processing competitions that have been held
over the years (VOLCANO09--nodule detection, EMPIRE10---motion
estimation, and LOLA11---lung and lobe segmentation, and
VESSEL12---vasculature segmentation) and it appears that the vast
majority of proposed algorithms are not publicly available. The reviewer
asserts that ``many of the feature indices (Table 2) already exist in
generic form in Brain packages, etc.'' We are unaware of such
availability in the major brain packages (e.g., FSL, SPM, FreeSurfer)
although we are aware that a significant number of them do exist in the
Insight Toolkit as we created the corresponding software and put them
there. Part of ITK-Lung would be targeting these features to lung
images.

\textbf{Naivety on the part of the investigators (Critique 2).} The
concern is raised regarding the necessary level of sophistication of the
investigators to fulfill the project deliverables. Although the
reviewers point out the software expertise of our team and our success
in the neuroimaging domain, it is assumed that we are ``too naïve with
respect to the difficulty and complexity of creating {[}targeted
pulmonary image analysis{]} software.'' Based on our publication record
in the field, we find this assessment to be a significant undervaluing
of our research contributions where much of our showcased
``preliminary'' software has been used in large studies. We thank the
reviewer for pointing this out and have added a section outlining these
publications.

\textbf{Reduction of algorithmic scope. (Critique 2)} Concerns
describing the scope of the project as being too large were made in
light of the extensive methodological depth of the field. We agree and
note that we have no intention of implementing the vast majority of
these methods. We have revised our application to sharpen the
delineation of our contributions. Additionally, we have dropped nodule
detection from the set of lung-specific algorithms targeted in this
application although such an implementation might be made available
during the course of our interaction with external collaborators.

\hypertarget{refs}{}

\end{document}
